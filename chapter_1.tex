%   Filename    : chapter_1.tex 
\chapter{Introduction}
\label{sec:researchdesc}    %labels help you reference sections of your document

\section{Overview}
\label{sec:overview}

Language is how humans communicate and express themselves \cite{Crystal_Robins_2024}.
It is dynamic because there are endless structural possibilities, changes in word meanings, and new words created \cite{Libretexts_2021}.
Slang is a great example of the dynamic nature of language. Slang is an informal language used by people in the same social group \cite{Fernández-Toro_2016}.
It serves social purposes: to identify a group's members, communicate informally, and oppose established authority \cite{McArthur_1998}. Slang is highly contextual and pervasive, even in non-standard English. \cite{Roth-Gordon2020}
Its figurative nature and how it twists the definitions of the words used in it make it hard for outsiders to understand \cite{Mattiello2005ThePO}. 

In recent years, the internet has become a significant medium for the evolution and spread of language, giving rise to `internet slang' \cite{Liu_Zhang_Li_2023}.
Internet slang is a collection of everyday language forms used by diverse groups online \cite{Barseghyan2014ONSA}.
Ujang et al. (2018, as cited in \cite{Sabri2020}) state that Internet slang is not easily understood by people outside the social group or people who are not fluent in the language where slang is used.
This phenomenon is particularly prominent among the younger generation \cite{Maulidiya_Wijaya_Mauren_Adha_Pandin_2021}, where they use it to communicate and interact with friends.

Needs Fix
Today, Generation Alpha is the youngest generation. Generation Alpha refers to people born between 2010 and 2025.
They were born into an era of rapid technological advancement, where digital devices and the internet are integral to their daily lives \cite{McCrindle2020}.
Generation Alpha is also called the first true digital natives \cite{Jukic_Skojo2021}.
They are expected to be the most ``technologically" skilled and most educated generation as they are the native speakers of the language of the Internet \cite{Prensky_2001b}.
According to the study \textit{Understanding Generation Alpha}, Generation Alpha is socially driven, which may let them grow up to be creative and unconventional, potentially shaping them to be assets in the future \cite{Jha2020}.

Since Generation Alpha was born with technology, the usage of Internet slang has been prominent in this generation.
However, it can create communication barriers between older and younger generations (Venter, 2017 as cited in \cite{Ghazali_Abdullah_2021}).
The communication barriers caused by the usage of Internet slang also affect people from the younger generation, especially individuals who are less active on social media and have less exposure to them \cite{Vacalares_Salas_Babac_Cagalawan_Calimpong_2023}.
This gap highlights the need for a tool that can bridge the generational divide, making it easier for individuals to understand the language of Generation Z.
By fostering a mutual understanding, such tools can promote more effective and harmonious interactions across generations, enhancing relationships and reducing miscommunication.

This study proposes the use and fine-tuning of Large Language Models and leveraging their extensive knowledge to develop a slang translation tool that can capture the nuances of Generation Z slang and produce a correct and coherent translation. This study is not only interested in the translation of the slang but with the creation of a framework that can be used as a basis for future LLM assisted translation work.

\section{Problem Statement}
\label{sec:problem_statement}

Internet slang fosters informal, relatable communication within the younger generation \cite{Ghazali_Abdullah_2021}, especially Generation Alpha, but it presents challenges in understanding for people outside this demographic. 
The gap in comprehension with older generations widens as internet slang evolves, often leading to miscommunication affecting social relationships that contribute to the generational divide \cite{Vacalares_Salas_Babac_Cagalawan_Calimpong_2023}. 
A more specific translation tool developed using language models can be used to bridge this divide.

By leveraging the ability of LLM to generate a more nuanced and properly constructed answer, a better tool can be made to translate the slangs into proper sentences.
It has already been proven by the likes of GPT being modified and tailored for use in several automated chatbots to provide customer service.
However, no such tool exists for slang translation of Generation Alpha, which arguably has the most diverse slangs compared to other generations.
The creation of this tool will allow translating of such texts into formal sentences and help with bridging the generational divide between them and older people, especially teachers. 

%\subsection{Research Objectives} %for improvement
%\subsubsection{General Objective}
\section{Research Objectives}
\label{sec:research_objectives}

\subsection{General Objectives}
\label{sec:general_objectives}
This study aims to fine-tune the zephyr-7b Large Language Model (LLM) for use in the translation of Generation Z internet slang used by Filipinos in social media.
%\subsubsection{Specific Objectives}
\subsection{Specific Objectives}
\label{sec:specific_objectives}
\begin{itemize}
	\item To create a dataset of sentences containing Generation Z slang used in differing contexts and its formal translation
	\item To create a Low Rank Adaptation (LoRA) implementation for fine-tuning an existing model
	\item To fine-tune an existing LLM to translate sentences containing Generation Z slang into formal sentences
	\item To evaluate the performance of the trained model and compare it to the based model using several performance metrics
\end{itemize}

%\subsection{Scope and Limitations of the Research} 
\section{Scope and Limitations of the Research}
\label{sec:scope}
This study will focus on the usage of internet slang by Filipino Generation Alpha, with an emphasis on English language since it is widely use on different digital platforms such as social media.

%\subsection{Significance of the Research}
\section{Significance of the Research}
\label{sec:significance}
The study contributes to understanding the evolving linguistic landscape shaped by internet slang, especially as used by Generation Alpha.
Insights gained from this study may aid educators, parents, and communication professionals in bridging inter-generational communication gaps and fostering better understanding across age groups.
