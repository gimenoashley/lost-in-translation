%   Filename    : abstract.tex 
\begin{abstract}
Internet slang is an informal variation of language that is prominent to the younger generation. The usage of this language brought miscommunication to older generations. This study focuses on Filipino Generation Alpha and their use of internet slang. This study aims to develop a translation tool leveraging Large Language Models (LLMs) to bridge this generational divide. A dataset of Generation Alpha slang sentences and their formal equivalents will be created, followed by the implementation of Low-Rank Adaptation (LoRA) to fine-tune an existing LLM. The model will be trained to translate slang sentences into formal English, and its performance will be evaluated against the baseline model using various performance metrics. The study highlights the significance of addressing communication gaps and provides insights into how technology can enhance understanding  and reduce miscommunications across generations. This research contributes to the broader discourse on language adaptation and generational communication in the digital age.


\begin{flushleft}
\begin{tabular}{lp{4.25in}}
\hspace{-0.5em}\textbf{Keywords:}\hspace{0.25em} & Internet Slang, Generation Alpha, Miscommunication, LoRA, LLM\\
\end{tabular}
\end{flushleft}
\end{abstract}
